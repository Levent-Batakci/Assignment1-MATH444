\documentclass{article}

\usepackage[pctex32]{graphics}
\usepackage{amssymb}
\usepackage{hyperref}
\hypersetup{
    colorlinks=true,
    linkcolor=blue,
    filecolor=magenta,      
    urlcolor=blue,
}
\usepackage[left=3cm,top=2cm,right=3cm,nohead,nofoot]{geometry}
\usepackage{latexsym}
\usepackage{epsfig}
\parindent0pt
\parskip6pt
\pagestyle{empty}
%\pagestyle{empty}
%\usepackage{fancybox}
\usepackage{pstcol}
\newcommand{\R}{{\mathbb R}}
\title{Math 444: Project 1}
\author{Levent Batakci}
\date{February 21, 2021} %Due Date
\begin{document}
\maketitle

The work in this project focuses on data visualization and basic clustering.
All of the mentioned implementations are coded in MATLAB and can be found in my \href{https://github.com/Levent-Batakci/Assignment1-MATH444}{public GitHub Repository}. 

\bigskip
{\Large {\bf Problem 1 - Model Reduction Data}}
\bigskip

The first data set consisted of data vectors $x^{(j)} \in \mathbb{R}^6$, with $1 \leq j \leq 4000$. 
The data is stored in the code as matrix $X\in \mathbb{R}^{6\times4000}$ and is loaded from the file \textit{ModelReductionData.mat}.
As an assessment of the raw data, we produced 2D scatter plots (Figure 1) corresponding to each pairing of components.

\textbf{INCLUDE FIG 1}

It's clear from the plots that there is clustering within the data.
Furthermore, there tentatively appear to be two clusters.

Next, we determined the data's effective dimesionality.
The first step I took was to center the data and then compute its Singular Value Decomposition (SVD).
To compute the centered data $X_c$, we subtract the mean $\bar{x}=\frac{1}{4000}\sum_{j=1}^{4000}x^{(j)}$ from each column of $X$.
The following code snippet performs these steps in MATLAB.
\begin{verbatim}
%Center the data
N = size(X,2);
xc = sum(X, 2)/N; %Get the average
Xc = X -  xc*ones(1,N); %Subtract out the average

[U,S,V] = svd(Xc, 'eco'); %Compute the SVD
\end{verbatim}

Once the SVD was computed, the singular values were extracted and plotted.
The resulting plot (Figure 2) shows that the first three singular values are significantly larger than the rest.
This leads to the conclusion that $X$ has an effective dimensionality of 3.

\textbf{INCLUDE FIG 2}

From there, we computed the matrix $Z_3$ of the first 3 principal components.
To do so, we used the first 3 feature vectors $u_1, u_2,$ and $u_3$ -- which are the first 3 three columns of the matrix $U$ from the SVD.
Specifically, we computed $Z_3 = U_3^TX$ where $U_3 = [u_1\;u_2\;u_3]$ in MATLAB.
\begin{verbatim}
U3 = U(:, 1:3)'; %Get the first 3 feature vectors
Z3 = U3 * Xc; %Project to get the 3 principal components
\end{verbatim}
 
After that, we compared the first three principal components with 2D scatter plots and a 3D scatter plot (figure 3).
From these graphics, it is clear that there are two clusters within the data.

\textbf{INCLUDE FIG 3}
 
\bigskip
{\Large {\bf Problem 2 - Biopsy Data}}
\bigskip


\bigskip
{\Large {\bf Problem 3 - Iris Data}}
\bigskip


\bigskip
{\Large {\bf Problem 4 - Aligned Face Data}}
\bigskip


\end{document}